\NeedsTeXFormat{LaTeX2e}[1996/06/01]
\documentclass[11pt,twoside,makeidx,hidelinks,]{memoir}
\usepackage{layouts}[2001/04/29]
\def\isfinal{1}

% Packages
%=========
\usepackage{verbatim}
\usepackage[colorlinks=false]{hyperref}
% \usepackage[utf8x]{inputenc}
\usepackage{listings}
\usepackage{
  fancyvrb,
  graphics,
  url,
  rotating,
  lipsum,
  microtype,
  fancybox,
  glossaries,
  titling,
  xspace,
}
\renewcommand*{\thefootnote}{\fnsymbol{footnote}}

% Boolean-ness
%=============
\newcommand{\onlyfinal}[1]{\ifdefined\isfinal{#1}\fi}%
\newcommand{\notfinal}[1]{\ifdefined\isfinal\else{#1}\fi}%
\newcommand{\whenfinal}[2]{%
  \ifdefined\isfinal%
    #1%
    \else%
    #2%
  \fi}%

%
% Page Layout
%============


\setstocksize{9in}{6in}
\setulmarginsandblock{0.8in}{0.5in}{*}
\settrims{0in}{0in}
\setlrmarginsandblock{0.8in}{0.5in}{*}
\settrimmedsize{9in}{6in}{*}
\settypeblocksize{7.7in}{4.6in}{*}

\setmarginnotes{0.1pt}{0.2in}{\onelineskip}
\setheadfoot{\onelineskip}{2\onelineskip}
\setheaderspaces{*}{2\onelineskip}{*}
\linespread{1.15}

\checkandfixthelayout

\setlength{\headsep}{8pt}

\newlength{\AfterFancyBreakMin}
\setlength{\AfterFancyBreakMin}{3\baselineskip}
\setlength{\parindent}{1.5 em}
\makeatletter
\g@addto@macro\quote{\linespread{1}\small{}}
\g@addto@macro\quotation{\linespread{1}\small{}}
\g@addto@macro\caption{\linespread{1}\footnotesize{}}
\makeatother
\tolerance 1414
\hbadness 1414
\setlength{\emergencystretch}{3em}
\righthyphenmin 62
\emergencystretch 1.5em
\hfuzz 0.3pt
\widowpenalty=10000
\clubpenalty=10000
\setlength{\parindent}{1.5 em}

% Page Heading
%-------------
\makepagestyle{ebook}
\makeoddhead{ebook}{}{\textsf\theauthor}{\normalfont\thepage}
\makeevenhead{ebook}{\normalfont\thepage}{\textsf \thetitle}{}

\aliaspagestyle{title}{empty}

% Text Tweaks
%-------------
\def\thinspace{\kern .16667em }
\newcommand{\ellip}{\ldots}
\newcommand{\Dash}{\thinspace---\thinspace}
\newcommand{\dash}{\tk{--Wrong Dash--}}

% Fonts
%------
\usepackage{fontspec}
\usepackage{xunicode}
\usepackage{xltxtra}
\defaultfontfeatures{Mapping=tex-text}
\setmainfont{Georgia}
\setsansfont{Tahoma}
\setmonofont{Courier}
\newfontfamily\scshape[Letters=SmallCaps, Numbers=Uppercase]{Courier}
\DeclareOldFontCommand{\tt}{\normalfont\ttfamily}{\mathtt}
\ifdefined\disableligature
  \DisableLigatures[f]{encoding = *, family = * }
\fi

% Dates
%-------
\usepackage{datetime}
\newdateformat{yearonly}{\THEYEAR}
\newdateformat{usmonthyear}{\monthname[\THEMONTH], \THEYEAR}
\newcommand{\monthyear}{%
  \ifcase\month\or January\or February\or March\or April\or May\or June\or
  July\or August\or September\or October\or November\or
  December\fi\space\number\year
}

% Document Headings
%==================

% Chapter
%-----------
\setsecheadstyle{\bfseries\large\centering}
\newcounter{renewsection}
\newcommand{\Chapter}[1]{
  \setcounter{renewsection}{\value{section}}
  
  \chapter{}
  
  \setcounter{section}{\value{renewsection}}
}
\chapterstyle{chappell}


% Section
%---------
\renewcommand{\thesection}{\arabic{section}.}

% Novel Tweaks
%=============

\newenvironment{xdedication}%
        {\vspace*{6ex}\begin{quotation}\begin{center}\begin{em}}%
        {\par\end{em}\end{center}\end{quotation}}%

\newenvironment{dedication}
  {\clearpage           % we want a new page
   \thispagestyle{empty}% no header and footer
   \vspace*{\stretch{1}}% some space at the top
   \itshape             % the text is in italics
   \raggedleft          % flush to the right margin
  }
  {\par % end the paragraph
   \vspace{\stretch{3}} % space at bottom is three times that at the top
   \clearpage           % finish off the page
  }

% Scene Formatting
%-----------------

\newcommand*\thoughtbreak{
  \begin{center}
    \vspace{1cm}
  \noindent\rule{0.5\textwidth}{0.4pt}
    \vspace{1cm}
  \end{center}
}

\newcommand{\thought}[1]{\emph{#1}}
\newcommand{\email}[1]{\texttt{#1}}
\newcommand{\status}[1]{\notfinal{\textbf{Status:} #1\newline}}
\newcommand{\synopsis}[1]{\notfinal{\line(1,0){320}\newline\textbf{What's going on here?} #1\newline
\line(1,0){320}\newline}}
\newcommand{\slug}[1]{\notfinal{\MakeUppercase{#1}\newline\newline}}
\newcommand{\slugline}[2]{\notfinal{\section{#1}}\onlyfinal{\section{}}\slug{#2}}
\newcommand{\interlude}[1]{\plainfancybreak{\AfterFancyBreakMin}{1}{#1}}

% \newcommand{\scene}[3]{%
%   \notfinal{\section{#1}\slug{#2}}
%   \input{#3}
%   \pbreak
% }%

\newcommand{\smee}[1]{\${-} \textsc{\small#1}}
\whenfinal%
  {\newcommand{\pbreak}{\pfbreak}}%
  {\newcommand{\pbreak}{\begin{center} \# \# \# \end{center}}}

% \renewcommand{\pbreak}{\thoughtbreak}
% Series Title Page
%==================
\def\seriestitle#1{\def\theseriestitle{\uppercase{#1}}}
\def\seriespage{\thispagestyle{empty}
\begin{center}
  \noindent{\large\theseriestitle}
  \vskip10.5pt
  \noindent Other Books in the series:
  \vskip4pt
  \theseriesbooks
\end{center}\cleardoublepage}
\def\seriesbook#1{\vskip.5pt{\noindent\hskip8pt\emph{#1}}\vskip1sp}
\def\seriesbooks#1\endseriesbooks{\def\theseriesbooks{\let\book\seriesbook{#1}}}

% Indices
%=========
\makeglossary
\makeindex
\let\oldindex\index{}%
\newcommand{\marked}[1]{\whenfinal{#1}{\underline{#1}}}%
\renewcommand{\index}[1]{\oldindex{#1}\marked{#1}}%
\newcommand{\Index}[2]{\oldindex{#1}\marked{#2}}%
\newcommand{\orbital}[1]{\oldindex{Orbital!#1}\marked{#1}}%
\newcommand{\equipment}[1]{\oldindex{Equipment!#1}\marked{#1}}%
\newcommand{\Equipment}[2]{\oldindex{Equipment!#1}\marked{#2}}%
\newcommand{\System}[2]{\oldindex{System!#1}\marked{#2}}%
\newcommand{\system}[1]{\oldindex{System!#1}\marked{#1}}%
\newcommand{\character}[1]{\oldindex{Character!#1}\marked{#1}}%
\newcommand{\Character}[2]{\oldindex{Character!#1}\marked{#2}}%
\newcommand{\ship}[1]{\oldindex{Ship!#1}\marked{\emph{#1}}}%
\newcommand{\Ship}[1]{\oldindex{Ship!#1}\marked{\emph{#1}}}%
\newcommand{\location}[1]{\oldindex{Location!#1}\marked{#1}}%
\newcommand{\Location}[2]{\oldindex{Location!#1}\marked{#2}}%
\newcommand{\clin}[1]{\oldindex{Clin!#1}Clin \marked{#1}}%

% Book-Specific Metadata
%========================

\def\NextBook{}
\def\ReturningCharacters{}
\title{Stranded Series Bible}
\author{Ben Wilson}
\date{2019-01-10}
\newdate{firstprint}{10}{01}{2019}


\seriestitle{Series TItle}
\seriesbooks
  \book{Book in Series}
  \book{Book in Series}
\endseriesbooks


% Drafting Help
%==============
\newcommand{\tk}[1]{%
  \ifdefined\isdraft%
      % \textbf{TK-#1}
      \todo{#1}\textbf{TK-#1}%
    \else%
      \textbf{TK-#1}%
      % \todo{#1}\textbf{TK-#1}
  \fi%
}%
\newcommand{\TK}[1]{\tk{BAD-TK--#1}}

% Document
%========================
\begin{document}

% Frontmatter
%-------------
\frontmatter
  % r.1 - Half-Title
  \begin{titlingpage}
    \pagestyle{empty}
    \begin{center}
    \vspace*{2in}

    \Huge\textbf{\textsf\thetitle}

    \vspace*{0.75in}

    \Large\textsf\theauthor

    \vspace*{\fill}
    \end{center}
  \end{titlingpage}

  % r.2 - Series Title Page
  \seriespage
 \newpage
  % r.3 Titlepage
  \begin{titlingpage}
    \pagestyle{empty}
    \begin{center}
    \vspace*{\fill}

    \HUGE\textbf{\textsf\thetitle}

    \vspace*{0.25in}
    \line(1,0){150}
    \vspace*{0.25in}

    \Large\textsf\theauthor

    \vspace*{\fill}

    \vspace*{\fill}
    \includegraphics[width=2in]{_images/logo.png}\\[0cm]
    %\hspace*{\fill}\textsf{Dausha}\hspace*{\fill}\newline%
    %\textsf{Publishing}
    \end{center}
  \end{titlingpage}

  % r.4 Copyright Page
  \vspace*{\fill}
  \pagestyle{empty}

  \par\noindent\emph{\thetitle}
  \newline

  

  \par\noindent This is a work of fiction. Names, characters, places and incidents are either
  the product of the author's imagination or are used fictitiously, and any resemblance to
  actual persons, living or dead, business establishments, events or locales is entirely
  coincidental.\newline
  

  \par\noindent\emph{Copyright \copyright{} 2019 Ben Wilson.}\newline

  
  \par\noindent Cover Design by Ben Wilson\newline
  

  
  \par\noindent Book Design by Ben Wilson\newline
  

  


  \par\noindent All rights reserved.\newline

  \par\noindent No part of this publication may be reproduced, stored in a retrieval system, posted on the Internet, or transmitted, in any form or by any means, electronic, mechanical, photocopying, recording, or otherwise, without prior written permission from the author. The only exception is by a reviewer, who may quote short excerpts in a review.\newline

  \par\noindent \theauthor
  \par\noindent Visit my website at \url{http://example.com}\newline

  \par\noindent Printed in the United States of America
  \newline

  \par\noindent\textit{First Printing, \usmonthyear\displaydate{firstprint}}
  \newline

  \par\noindent ISBN-13 9-87654-3231-0123
  \vspace*{\fill}

  % r.5 - Dedication
  
  


  \newpage
\mainmatter
  \pagestyle{ebook}
  \sloppy

\chapter{Getting Started}\hypertarget{getting-started}{}\label{getting-started}

Welcome to Verku! Verku is the esperanto command to compose/write, and this little gem of an application is intended to help you do just that. This short guide is designed for you as a beginner to get started with Verku.

Verku tries to get styling out of your way so you can just write. To accomplish this, it uses \href{https://daringfireball.net/projects/markdown/}{Markdown} (specifically \href{kramdown.gettalong.org/syntax.html}{Kramdown}) to give you just enough formatting to write. It then uses a LaTeX variant to create a PDF suitable for printing; or HTML, ePUB and Mobi for electronic distribution.

First you will need to install some prerequisites:

\begin{itemize}
\item{} The \href{http://ruby-lang.org}{Ruby} interpreter version 2.0.0 or greater.
\item{} The \href{hhttps://en.wikipedia.org/wiki/XeTeX}{XeTeX} typesetting engine.
\item{} The \href{http://www.amazon.com/gp/feature.html?docId=1000765211}{KindleGen} converter.
\end{itemize}

\section{Installing Ruby}\hypertarget{installing-ruby}{}\label{installing-ruby}

To install Ruby, consider using \href{http://rvm.io}{RVM} or \href{http://rbenv.org}{rbenv}, both available for Mac OSX and Linux distros. If you're running a Windows, well, I can't help you. I don't even know if Verku runs over Windows boxes, so if you find any bugs, make sure you \href{https://github.com/Merovex/verku/issues}{let me know}.

\section{Installing XeTeX}\hypertarget{installing-xetex}{}\label{installing-xetex}

\href{hhttps://en.wikipedia.org/wiki/XeTeX}{XeTeX} is a TeX typesetting engine using Unicode and supporting modern font technologies such as OpenType, Graphite and Apple Advanced Typography (AAT). We're using XeTeX because TeX is one of the best ways of formatting a beautiful hard-copy book.

\begin{itemize}
\item{} On Mac: To install on a Mac, you will want to install the \href{https://tug.org/mactex/}{MacTEX distribution}.
\end{itemize}

\section{Installing KindleGen}\hypertarget{installing-kindlegen}{}\label{installing-kindlegen}

KindleGen is the command-line tool that allows you to convert e-pubs into \texttt{.mobi} files. Once you've done that, then you can make your work available via \href{https://www.createspace.com/pub/member.dashboard.do}{CreateSpace}.

If you're running \href{http://brew.sh}{Homebrew} on the Mac OSX, you can install it with \texttt{brew install kindlegen}. Go to \href{http://www.amazon.com/gp/feature.html?docId=1000765211}{KindleGen's website} and download the appropriate installer otherwise.



\section{Creating Chapters}\hypertarget{creating-chapters}{}\label{creating-chapters}

You can create chapters by having multiple files or directories. They're alphabetically sorted, so make sure you use a prefixed file name like  \texttt{01\_Introduction.md} as the file name.

If you're going to write a long book, make sure you use the directory organization. This way you can have smaller text files, which will be easier to read and change as you go. A file structure suggestion for a book about \href{http://guides.rubyonrails.com}{Ruby on Rails} would be:

\texttt{text
getting-started-with-rails
├── text
    └── 01\_Guide\_Assumptions.md
    └── 02\_Whats\_Rails.md
    └── 03\_Creating\_A\_New\_Project
        └── 01\_Installing\_Rails.md
        └── 02\_Creating\_The\_Blog\_Application.md
    └── 04\_Hello\_Rails
        └── 01\_Starting\_Up\_The\_Web\_Server.md
        └── 02\_Say\_Hello\_Rails.md
        └── 03\_Setting\_The\_Application\_Home\_Page.md
    └── ...
}

Notice that the file name does not need to be readable, but it will make your life easier.



\chapter{Generating output}\hypertarget{generating-output}{}\label{generating-output}



\chapter{An h1 header}\hypertarget{an-h1-header}{}\label{an-h1-header}

Paragraphs are separated by a blank line.

2nd paragraph. \emph{Italic}, \textbf{bold}, and \texttt{monospace}. Itemized lists
look like:

\begin{itemize}
\item{} this one
\item{} that one
\item{} the other one
\end{itemize}

Note that --- not considering the asterisk --- the actual text
content starts at 4-columns in.

\begin{quotation}
Block quotes are
written like so.

They can span multiple paragraphs,
if you like.
\end{quotation}

Use 3 dashes for an em-dash. Use 2 dashes for ranges (ex., ``it's all
in chapters 12--14''). Three dots \ldots{} will be converted to an ellipsis.
Unicode is supported. ☺

\section{An h2 header}\hypertarget{an-h2-header}{}\label{an-h2-header}

Here's a numbered list:

\begin{enumerate}
\item{} first item
\item{} second item
\item{} third item
\end{enumerate}

Note again how the actual text starts at 4 columns in (4 characters
from the left side). Here's a code sample:

\begin{verbatim}# Let me re-iterate ...
for i in 1 .. 10 { do-something(i) }
\end{verbatim}

As you probably guessed, indented 4 spaces. By the way, instead of
indenting the block, you can use delimited blocks, if you like:

\begin{verbatim}define foobar() {
    print "Welcome to flavor country!";
}
\end{verbatim}

(which makes copying \& pasting easier). You can optionally mark the
delimited block for Pandoc to syntax highlight it:

\begin{lstlisting}[showspaces=false,showtabs=false,language=python,basicstyle=\ttfamily\footnotesize,columns=fixed,frame=tlbr]
import time
# Quick, count to ten!
for i in range(10):
    # (but not *too* quick)
    time.sleep(0.5)
    print i

\end{lstlisting}   %  class="language-python"

\subsection{An h3 header}\hypertarget{an-h3-header}{}\label{an-h3-header}

Now a nested list:

\begin{enumerate}
\item{} First, get these ingredients:

\begin{itemize}
\item{} carrots
\item{} celery
\item{} lentils
\end{itemize}
\item{} Boil some water.
\item{} Dump everything in the pot and follow
this algorithm:

\begin{verbatim}find wooden spoon
uncover pot
stir
cover pot
balance wooden spoon precariously on pot handle
wait 10 minutes
goto first step (or shut off burner when done)
\end{verbatim}

Do not bump wooden spoon or it will fall.
\end{enumerate}

Notice again how text always lines up on 4-space indents (including
that last line which continues item 3 above).

Here's a link to \href{http://foo.bar}{a website}, to a \href{local-doc.html}{local
doc}, and to a \hyperlink{an-h2-header}{section heading in the current
doc}. Here's a footnote \footnote{Footnote text goes here.}.

Tables can look like this:

size  material      color
----  ------------  ------------
9     leather       brown
10    hemp canvas   natural
11    glass         transparent

Table: Shoes, their sizes, and what they're made of

(The above is the caption for the table.) Pandoc also supports
multi-line tables:

\pfbreak
keyword   text
--------  -----------------------
red       Sunsets, apples, and
          other red or reddish
          things.

green     Leaves, grass, frogs
          and other things it's
          not easy being.
--------  -----------------------

A horizontal rule follows.

\pfbreak

Here's a definition list:

\begin{description}
\item[apples] Good for making applesauce.
oranges



Citrus!
tomatoes



There's no ``e'' in tomatoe.
\end{description}

Again, text is indented 4 spaces. (Put a blank line between each
term/definition pair to spread things out more.)

Here's a ``line block'':

\begin{longtable}{|l|}
\hline
Line one\\
Line too\\
Line tree\\
\hline
\end{longtable}

and images can be specified like so:

\begin{figure}
\begin{center}
\includegraphics{example-image.jpg}
\end{center}
\caption{example image}

\end{figure}

Inline math equations go in like so: \$\textbackslash{}omega = d\textbackslash{}phi / dt\$. Display
math should get its own line and be put in in double-dollarsigns:

\begin{displaymath}
I = \int \rho R^{2} dV
\end{displaymath}

And note that you can backslash-escape any punctuation characters
which you wish to be displayed literally, ex.: `foo`, *bar*, etc.



% Back Matter
%-------------
\backmatter

%% Bibliography
% \bibliographystyle{\mybibliostyle}
% \bibliocommand
% \pagestyle{empty}
\begin{center}
  \vspace*{\fill}

  {\Large \par
  \noindent Thank you for reading

  \vspace*{0.125in}
  {\LARGE\textbf{\textsf{\thetitle}}}
  \vspace*{0.125in}

  \ifdefined\ReturningCharacters
    \par
    \noindent\ReturningCharacters{} will return in \emph{\NextBook}.} \vspace*{\fill}
  \fi

  {\LARGE\textbf{\textsf{Follow Me}}}

  \vspace*{0.125in}

  \textbf{\theauthor{}} can be followed on Twitter and Facebook.
  \newline Visit his web site at http://dausha.net for more information. You can also sign up for announcements of other books in this series at http://dausha.net.

  \vspace*{\fill}
\end{center}



\end{document}
